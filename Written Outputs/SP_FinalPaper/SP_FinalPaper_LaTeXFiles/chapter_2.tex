%   Filename    : chapter_2.tex 
\chapter{Review of Related Literature}
\label{sec:relatedlit}

\section{Missing Persons Reporting and Finder Applications}

The paper written by Desale H., Tavasalkar P., Vare S., and Shintre R. titled “Android Crime reporter and Missing Person Finder”, details some key insights that can be extremely helpful for the development of the project \cite{desale2020android}. As the title suggests, the study proposed an application that allows users in India to directly report not only missing persons cases but also other crimes to the authorities through the app, with the goal of hastening the reporting process and, subsequently, solving the crime. 

Moreover,  their application features a “panic" feature to quickly record a crime and an "alert" feature to immediately inform authorities of wanted criminal sightings. The proposed app also allowed the user to get updates regarding their submitted case, and allowed authorities to track cases more conveniently.  This aspect of the application is directly in line with the goals of the study and is proof of the viability and significance of the HanApp framework.

One key difference with this application compared to HanApp, however, is that this only sends case reports and notifications directly to the police with the sole purpose of expediting the filing of these cases. Whereas HanApp's framework expanded the scope by enabling and enforcing the dissemination of verified information with regards to missing persons cases.

\section{Mobile Applications for Alerting the Community Regarding Missing Persons}

A mobile alert application determined to engage community volunteers to help in locating missing persons with dementia called the “Community ASAP system” was developed and documented in a paper by Neubauer, et. al. (2021). The findings of the study’s simulation of the Community ASAP system highlighted the importance of police services in these cases on account of their primary and direct involvement, and the effectiveness of community response and participation to the occurrence of missing persons with dementia. The approach also proves to be viable even for people who have no social media which is popular for being an accessible medium to disseminate information regarding missing persons \cite{neubauer2021mobile}.

“CoSMiC”, a mobile application designed to crowdsource information about lost and missing children in situ is another approach towards applying techniques and technologies in locating missing persons. The main concern that the CoSMiC mobile application prioritizes to solve is with regards to the urgency and criticality that ensues whenever a child goes missing within a neighborhood. The application aims to digitize crowdsourcing for finding missing children through a landmark-based location history of the lost child that was chronologically and locationally procured within the network of crowdsourced information \cite{shin2014cosmic}.

The urgency, reasoning, and crowdsourcing proposition and concerns by the above studies also reflect one of the main aims of the application being proposed; the alerting and disseminating of information regarding missing persons in order to encourage and widen the scope of community participation in resolving missing persons cases.

\section{Serverless Application Model}

Serverless does not mean “server-less”. Serverless Computing is an up-and-coming paradigm and framework for the development and deployment of multiple applications, now relying on services through the cloud. A recent shift of enterprise applications’ architectures into containers, microservices, and serverless backend services has pushed the paradigm even more. Serverless platforms offer new features that make creating scalable microservices and applications easier and more cost efficient, promoting themselves as the next stage in cloud computing architectural evolution \cite{castro2017serverless}. One example of an application development software used in a serverless framework is Google’s Firebase, an example of BaaS (Backend-as-a-Service).

The application suite, HanApp, also utilizes serverless computing as its paradigm. Knowing the general scenarios, perception, and support towards this new paradigm in software development is essential as it helped in the decision-making of the developers during the development of the application suite.

\subsection{Google Firebase}

As defined by Khawas, C. and Shah, P. in their study titled “Application of Firebase in Android App Development-A Study” (2018), Firebase is one of the relatively new and even faster approaches towards handling large amounts of unstructured data as compared to the traditional Relational Database Management Systems (RDBMS) through developing serverless applications 
\cite{khawas2018application}.

In a paper by Hannula, T. (2021) titled “Unity mobile application with a serverless Firebase backend”, where a lo-fi themed Android mobile application prototype was developed, both NoSQL database services that are Firebase Realtime Database and Cloud Firestore were both utilized in managing the backend of the application the author has developed. Both services have enabled the application to be simplified and streamlined as Firebase handles the maintenance of the data in the Realtime Database and Cloud Storage \cite{hannula2021unity}.

As stated by the definition and example above, Google Firebase, an approach for serverless computing, was a promising choice for the chosen serverless framework for HanApp. Knowing the benefits and how Firebase was implemented on mobile applications was crucial in the development of HanApp.


\section{Location Services in Missing Persons Applications}

\subsection{Global Positioning System (GPS)}

The Global Positioning System (GPS) is, in itself, a United States-owned service that offers positioning, navigation, \& timing (PNT) services to its users. As it is free, open, and reliable, GPS has been used and integrated countless times on a myriad of applications, including ones that are implemented in mobile platforms \cite{gpsGov}.

A seamless application of the Global Positioning System to Android mobile phones was implemented in a paper titled “Abhaya: An Android App for the Safety of Women" by Yarrabothu, R. S., and Thota, B. (2015). The paper describes an app called “Abhaya”, where it employs a quick and easy-to-use alert button from the application integrated within the smartphone, in which a single tap can identify the location or place of the user through the use of GPS services and would thus then send a message comprising the location URL to all registered contacts of the user. Additionally, Abhaya also calls the first person in the registered contact list of the user and periodically sends an SMS every five minutes to the registered contacts until the stop button has been tapped \cite{yarrabothu2015abhaya}.

A paper titled “Mobile phone application for reporting and tracking missing persons in Kenya ”, by Elizabeth Mutisya for the Strathmore University (2017), describes developing a centralized system and a mobile application, that hopes to organize the rather inefficient process in Kenya with regards to addressing missing persons cases. The said application uses GPS to determine and tag the location of a missing person, given that missing person has the application on hand, as well as of the sightings reported to increase the effectiveness of the app. It also utilizes the currently centralized database, National Missing and Unidentified Persons System (NamUs) in Kenya to both counter-check, verify, and assist in finding these said missing persons. A web application was also developed in order to assist those who would still want to access the software but do not have the smartphone needed for it to run the application (Android). The application was developed incrementally, starting on a smaller scale and gradually increasing in complexity using the Agile methodology \cite{mutisya2017mobile}.

%Mutisya, E. (2017) has proposed the utilization of the Global Positioning System (GPS) in the assistance in providing more details about the disappearance of a missing person. This approach has exemplified the fact that GPS services can be applied in finding missing persons. The Abhaya application \cite{yarrabothu2015abhaya} eventually provided the option for continuous tracking of the user or the missing person, provided that he or she has enabled the GPS in the proposed application, as periodically sending SMS instead of real-time tracking requires less processing power compared to the latter.

Overall, the utilization of the Global Positioning System (GPS) has provided an exemplary mechanism for tracking down missing persons by either identifying the last location the person was found or a real-time continuous tracking. This employs heightened identification of an individual’s whereabouts thus expediting the searching process.

\subsection{Google Maps Platform}

There have been numerous online mapping services available to the public, arguably the most popular of which is Google Maps. Google Maps, a web mapping service by Google, provides satellite imagery, street maps, real-time traffic conditions, and also route planning \cite{antony_2021}. Its closest competitor, Bing Maps, was developed by Microsoft, and a study showed that both provided near-similar accuracy in geocoding \cite{kilic2020accuracy}. However, due to the fact that Google Maps is more feature-rich, and has over 1 billion active users monthly — which translates to frequent location mapping and verification, Google Maps is considered to be the better option \cite{lookingbill2019google}.

Furthermore, Google has made it possible for developers to integrate and utilize Google Maps in the development of applications that require maps and location data. This is offered through the Google Maps Platform which is a collection of APIs, SDKs, and tools to easily embed and allow data retrieval from Google Maps to applications \cite{googleDevelopers}. In fact, there have been numerous studies showing the application of Google Maps and their open APIs. 

In Ghana, Google Maps together with GPS was used to successfully map accurate digital postal addresses to a separate application (GhanaPostGPS) by overlaying it over Google Maps through their API \cite{gah2018using}. Another study utilized GPS, Google Maps, and GSM technology (for texting location details) for parents and school authorities to accurately monitor children’s location in a timely manner which could be valuable if they go missing \cite{sunehra2016children}.

Google Maps and the Google Maps Platform’s open APIs and SDKs, together with GPS technology, makes it a viable technology to be used for developing apps for locating MPs in the country.

\subsection{Location based notification and alarm applications}

Technology and tools such as GPS and Google Maps (and the Google Maps Platform APIs), has made it possible to create applications to notify or alarm the user when they’re at (or near) certain locations. In 2013, an Android application was developed using Google Places API (offered in the Google Maps Platform) and GPS called “GEO ALERT” which would alert a traveler when they’re at a certain spot and it would show them a history of that specific location \cite{garg2013geo}. Another study proposed a location-based notification and alarm Android application utilizing GPS and LBS (location-based service), and it notified the user if a friend is nearby and alarms when the user enters a marked location on the map \cite{kanfade2018location}. More recently in 2022, a study titled “Travellert: A Location Based Alarm Application” developed an application using GPS and Google Maps Platform APIs and GPS which alerted a traveler when they were near their destination to avoid missing their stops \cite{travellert}.

The existence of these studies and location-based notification and alarm applications provide proof of feasibility in using GPS, Google Maps, and similar technologies on how to approach the development of the app’s location-based notification system for alerting users of PNP-verified MP cases near them.

\subsection{Simplified User Experience and Interface Design for Crisis and Emergency-related Applications}

User experience (UX) design is critical in developing effective methods for locating missing people. When someone goes missing, their loved ones may go through a painful and difficult situation. UX design may assist in the development of tools and systems that make the search process simpler, more efficient, and less overwhelming. Hence, in addition to psychological and behavioral behaviors, the emotional experience of criminal behavior must be understood. There is an occurrence of the circumplex of emotions, which presents emotions in a circular order based on the aspects of arousal/non-arousal and pleasure/displeasure \cite{hunt_2020, suchana_2021}. Following this premise, the app will relatively be designed in a calm approach in order to mitigate individuals' emotional experiences during distress. The emotional needs of the users are the top-most priority and need to be taken into account for better user experience. 
 
Designing intuitive and simple-to-use search systems is an important part of UX design in missing persons instances. Everyone should be able to use the platform, including those who are not tech-savvy or have low resources. Illustrations are critical in increasing the user experience of an app \cite{suchana_2021}. They can be used to convey important information, improve the app's aesthetic appeal, and give a feeling of consistency and identity. With that said, the outlined and rounded features of the material design hinges on the notion to help customers through the search process, thus, the design is basic, with clear directions and user-friendly navigation. Also, the color scheme follows a high-contrast theme that helps critical information stand out and be more easily identified. Indigo, a borderline shade of blue and purple, exudes a personality that is calm, trustworthy, and approachable in times of crisis which often results in reliability \cite{babich_2017}. Apart from that, this also caters users who are visually-challenged, especially color-blind users, to interact with the app without hassle. Deuteranomaly and Protanomaly are the common types of colorblindedness that impedes most people \cite{babich_2017, NEI_2019}. Both of them make an individual unable to tell the difference between red and green, hence indigo was utilized to recuperate their needs. 
 
Furthermore, the app also focuses on the usability framework that leverages user experiences. Since the app focuses on a distress manner, the application expects to perform smoothly without consuming excessive phone resources and the content relevancy is determined by the app's goal and the information's proximity to the time and location of the hazard event. Tan, et.al. (2020) established the human-computer interaction frameworks that best suit in times of crisis which can be advantageous for user retention and usability. As a result, another crucial concern in missing persons UX design is ensuring that the platform provides useful and up-to-date information \cite{tan_etal_2020}. This information comprises the missing person's physical description, last known location, and any identifying features. It also contains information about the search activities, such as progress, fresh leads, and how individuals can help with the search.
