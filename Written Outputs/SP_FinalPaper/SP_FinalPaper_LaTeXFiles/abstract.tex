%   Filename    : abstract.tex 
\begin{abstract}

Incidents of people going missing is an ongoing problem in the country. Although missing persons cases typically occur after natural or man-made disasters, estimates and reports indicate that a sizable number of Missing Persons instances still happen outside of these settings. The Philippine National Police (PNP) has developed measures to address this issue, however the established reporting system has neither taken advantage of recent technology to streamline the process nor leveraged a community-based approach to improve the search efforts.

This research aims to address both the main issue of missing persons in the country and the need to improve the reporting system by developing an application suite that: streamlines the missing persons case reporting, management, and verification process, and enables community-based search efforts.

The application suite, “HanApp”, which is comprised of two application interfaces (the main user application and the PNP application) is developed using Flutter Framework for cross-platform compatibility, and utilizes Flutter development packages, Google Location Services, Google Maps Platform APIs, and Firebase serverless framework, database, and storage to achieve these features. 

Results show that the main user application enables users to sign up for an account, report missing persons cases, receive updates on the status of their submitted reports, as well as get notified and view PNP-verified missing persons cases in their vicinity. Additionally, the PNP application enables PNP accounts to receive, manage, and verify missing persons cases that occur within their station’s area of jurisdiction.


\begin{flushleft}
\begin{tabular}{lp{4.25in}}
\hspace{-0.5em}\textbf{Keywords:}\hspace{0.25em} & missing persons, case reporting and verification, community-based search, location tracking\\
\end{tabular}
\end{flushleft}
\end{abstract}
