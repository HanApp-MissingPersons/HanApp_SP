%   Filename    : chapter_1.tex 
\chapter{Introduction}
\label{sec:researchdesc}    %labels help you reference sections of your document

\section{Overview of the Current State of Technology}
\label{sec:overview}
As defined by the International Commission on Missing Persons (ICMP), subjectively, anyone who is being sought by at least another person and whose location or whereabouts are unknown can be classified as a missing person \cite{icmpMissing}. However, each country has their own standard or policies for legally defining a "missing" person, and as such, accurate statistics on the average rate of missing persons globally are harder to discern. This, together with the unconfirmed number of the unreported cases of missing persons, people who voluntarily go missing, or even victims of disasters and conflict, brings to light how obscure and challenging a search for a missing person can be.

With this, there have been efforts globally to implement technologies and systems for organizing and solving this problem. The AMBER (America's Missing Broadcast Emergency Response) alert system, while not a standalone application, is a system that geo-locates and uses various forms of effective media such as smartphones, television, or radio to disseminate information about missing or abducted children in the United States in order to elicit citizen tips and responses in order to expedite the rescue of the said missing and/or abducted children \cite{griffin2007preliminary}.

NamUs, or the National Missing and Unidentified Persons System in the United States, is an integrated set of two databases: one for the information about unrecorded persons whose remains are inside the United States, and another for all profiles of missing persons and their information. Being publicly accessible for all entries and searches, the NamUs database has contributed to multiple projects, studies, works, and investigations for law enforcement \cite{murray2018history}. 

A push for a national Missing and Found Person Database (MFPD) in a 2016 Memorandum Circular from the National Police Commission in the Philippines has defined the database as a repository of all the names and relevant information about reported missing and found persons in the country. A website called the “Missing Person Website” was also defined in the memorandum, with the purpose of posting and displaying the name, picture, and other relevant information about missing and found persons \cite{NationalPoliceCommission}. The said database, however, is not open to the public unlike the NamUs. With further probing and scouring, the researchers have found that there is no clear, standardized, and modernized level of technology applied in the filing, dissemination, and searching for missing persons in the Philippines.

Currently, the standard procedure for filing missing persons (MP) cases in the Philippines is through the desk officers in their respective local police offices. This process can be cumbersome and counter-intuitive, particularly in cases wherein the missing person is a child or a victim of an accident in which the case recording, investigation, and monitoring need to begin immediately \cite{NationalPoliceCommission}.

In response to this, the researchers aim to develop an application suite that leverages current technology to streamline the processes required in the filing, managing, and verifying of missing persons cases. The application suite is modeled after the AMBER alert system, such that it uses location and map services in both pushing area-wide alerts or notifications and also in tagging the missing person's last seen location on a map. As such, the application suite enables community cooperation in the search for missing persons through the dissemination of PNP-verified missing persons reports. 


\section{Problem Statement}
The Philippine Statistics Authority (2021) have only reported missing persons statistics for cases involving natural and man-made disasters: namely 5,000 MPs in 2020 and 12,000 MPs in 2021. \cite{PSAOpenStat}. On the other hand, the Philippine News Agency (2021) shared data from the DOJ which reported over 1.2 million reports of missing and/or exploited children in 2020, and 2.8 million reports in 2021 \cite{pulta_2021}. These, unfortunately, are only among the few publicly available numbers provided by national agencies with regards to MP cases in the Philippines.

Orion Support Incorporated (2015) estimated 35,000 MP reports each year in the country. With this, it relatively coincides with the rate of 1.7 persons per 1,000 Filipinos \cite{orion_2021}. However, the obscurity of reporting these missing persons cases has troubled not only the community but also authorities to develop effective measures and guidelines to quickly act on missing persons cases.

The dire situation of these missing persons cases causes distress to their beloved in which our culture is deeply ingrained to familiarity and closeness of kinship or other relationships in the same nature.  Moreover, unresolved MP cases in the Philippines have drastic effects on the socio-economic health of the community, affecting not only the MP’s immediate family but the missing individual’s network and area. As MP cases remain unresolved, the sense of security in the area as well as the competence of local authorities are put into question, causing friction and conflict in the community. 

Although the PNP adopted MC 2016-033 which established a unified Missing and Found Persons Database (MFPD) and the process for receiving and handling MP reports in order to handle this issue, these guidelines stipulated that MP case reporting can only be done in person in the station of the area where the disappearance occurred and required the user to fill out numerous forms \cite{NationalPoliceCommission}. Requiring the MP case reportee to travel and file the case in-person takes away valuable time, and neglects to utilize existing technology to streamline the process and have them receive, verify, and act on the case immediately. Moreover, the PNP has advised the public against posting and sharing of unverified MP cases on social media \cite{madarang_2022}.

As such, there is a need for a system that leverages the use of existing technology to streamline and accelerate the reporting and verification process, and notify users of PNP-verified missing persons cases in their area. The creation of the missing persons reports and alerts application suite, "HanApp", addresses the aforementioned issues by enabling users to file and receive updates on missing persons cases, get alerts and view PNP-verified missing persons cases, and also allowing the PNP to receive, manage, and verify missing persons reports online.

\section{Research Objectives}
\label{sec:researchobjectives}

\subsection{General Objective}
\label{sec:generalobjective}

The general objective of this study is to develop an application suite with mobile and desktop interfaces for users who want to report and get updates on missing persons cases, users who want to be altered by and view verified missing persons cases in their area, and the PNP in order to expedite and simplify the filing, managing, verifying, disseminating, and resolving of missing persons cases. The application will be called "HanApp", a portmanteau of ``Hanap" (Tagalog for ``find") and App (for ``application"). 

\subsection{Specific Objectives}
\label{sec:specificobjectives}

This study specifically seeks to:

\begin{enumerate}
   \item Develop an application that would allow users to easily report missing persons to the PNP and receive updates on such reports, as well as a PNP counterpart application that will enable the PNP to quickly receive, verify, manage, and respond to reports.
   \item Integrate a serverless database system which securely stores and manages User and PNP accounts, login authorizations, and reports without the need for a server infrastructure or dependence on local data storage, allowing the application suite to be used in any operating-system compatible device.
   \item Implement geolocation services that allow for report routing so that reports are received and managed by the corresponding PNP account of the case’s jurisdiction. 
   \item Implement location-based notifications so that users can receive alerts and view information about missing persons near their current location so they can assist in the search, thus promoting a community-based approach in resolving missing persons cases.
\end{enumerate}

\section{Definition of Terms and Acronyms}

\textbf{Child} - refers to any person under the age of 18.

\textbf{Absent Person} - according to the PNP guidelines, absent persons are defined as those who are not in their domicile or place where they are supposed to be present in less than 24 hours, and whose families, relatives, or significant others have no clue as to their whereabouts but there is no apparent risk and do not require police investigation.

\textbf{Absent/Missing Person (\textbf{A/MP}) }- in accordance with the PNP guidelines and the affixed checklist when receiving and processing missing person cases, the term (A/MP) can refer to the reported person whose whereabouts is unknown and whose classification as Absent or Missing is not yet finalized.

\textbf{Database} - - also referred to as “\textbf{serverless database}”, refers to the application’s own public database containing the reports filed to the PNP pending their verification, the PNP-verified missing persons reports which will be used to notify nearby users of missing persons cases in their area.

\textbf{HanApp} -  also referred to as “\textbf{the application suite}”, refers to the set of applications and its various interfaces developed in this study. The interfaces include:
\begin{itemize}
    \item \textbf{Main (or User) App} - refers to the main application that has the features: reporting and getting updates on missing persons cases, receive alerts/notifications and view PNP-verified missing persons cases near their current location
    \item \textbf{PNP Admin App} - also referred to as “\textbf{PNP Application/App}”, refers to the application interface that is accessed by the PNP in which they receive, manage, verify, and provide updates on received missing persons reports.
\end{itemize}

\textbf{Located/Found} - will be used to denote that the missing person has been identified and reunited with their family or guardian.

\textbf{Missing person (MP)} - will refer to any person who is classified under the PNP guidelines as missing; an adult person who is missing is required under the guidelines to have not been located after 24 hours, whereas a child that has gone missing is immediately referred to as an MP.

\textbf{Missing and Found Person Database (MFPD)} - refers to the PNP’s own private database where they manage Missing and Found Persons cases, and will not be accessible by the application suite.

\textbf{PNP-verified report} - refers to the report filed through the Main/User App that has been verified as complete and legitimate by the PNP precinct personnel through the PNP App.

\textbf{Reportee} - refers to the user filing the missing person CASE report via the main application, and is able to view the status of their reports through the main user application.

\section{Scope and Limitations of the Research}
\label{sec:scopelimitations}

HanApp primarily aims to streamline and hasten the filing and verification of missing persons reports to the PNP and create an intuitive platform for the alerting and dissemination of PNP-verified missing persons reports.

HanApp covers the following classifications of missing persons in accordance to the PNP Guidelines: absent person that has not been located 24 hours from their perceived disappearance, missing children, missing victim of natural calamities and human-induced disasters/accidents, and missing persons believed to be a victim of violence and crimes.

Furthermore, HanApp employs a serverless approach in its accounts and data management through Firebase, as well as online services such as Google Maps and other location based services, all of which rely on stable internet connectivity, be it through mobile data or Wi-Fi. As a result, HanApp is an exclusively online application.

Finally, in order to facilitate report filing and management, as well as alerting and disseminating PNP-verified missing persons reports, a serverless database powered by Firebase is used to store the missing persons reports. It should be noted that this is a separate and different database from the PNP’s own “Missing and Found Persons” database (MFPD) and is exclusive to the application suite.

\subsection{User and Information Access}
\label{sec:userInfoAcccess}
The application will only have three interfaces which will cater to its two primary users — the general user, and the PNP personnel/helpdesk in charge of missing persons cases. Since this is an online-only application suite with a mobile interface for the main users and a desktop interface for the PNP, and the application’s scope is limited to the Philippines, only users in the Philippines with compatible smartphones and access to the internet will be able to use the app.

Due to the sensitivity of the data being handled, it’s important to define the scope and limits of each user’s access to address data privacy and security concerns.

\begin{enumerate}
    \item The general users refer to the general public who either wish to report a missing persons case, or assist with search efforts for missing persons. They will only have access to information regarding their registered account, the updates provided by PNP on the report they filed, and the information included in the PNP-verified missing persons in their area.
    \item The PNP personnel/helpdesk in charge of missing persons cases and have access to their HanApp PNP account will only have access to information provided through reports, and will not be given access to any other information from the users or their devices.
\end{enumerate}

Moreover, the application’s administrators and developers will only have access to the serverless database containing the submitted reports and the (PNP and User) account names and emails at most. As such, user passwords and user’s current location information will not be accessible by anyone apart from the users themselves. 

The PNP’s own private “Missing and Found Person Database” and all information therein will be separate from the serverless database to be used for storing missing persons reports, and will not be accessed by the application, the application’s users, nor its administrators.

\section{Significance of the Research}
\label{sec:significance}
The research’s application suite and its framework aim to consolidate and streamline the process of filing, managing, and verifying missing persons reports, and also allow for an efficient way to disseminate PNP-verified missing persons reports. Thus, accelerating the process of resolving missing persons cases through quicker report verification, location-based alerts, and leveraging community-based search efforts. 

This application and framework will be of great significance to affected individuals whose loved ones have gone missing by allowing them to report and ultimately locate the whereabouts of these missing individuals with ease.

Moreover, the application could serve as a great tool to the PNP by making it faster and more convenient to receive, manage, and verify missing persons reports, thus allowing them to act faster on missing persons cases while simultaneously reducing the propagation of unverified reports. In addition, the data that will be generated from the application suite’s usage by the public and the PNP can be used as reference regarding the statistics relating to missing persons in the country. 

The focal point of this research paves the way for a systematic and technological breakthrough here in the Philippines through the development of an application suite that could potentially assist in resolving missing person cases by hastening the reporting and dissemination process.

Lastly, the novelty of this study could potentially provide a baseline framework for similar studies impacting the creation of not only a more robust system for reporting missing persons cases in the country, but also the modernization of crime/incident reporting to the PNP through the use of online software rather than exclusively-in-person reporting.





