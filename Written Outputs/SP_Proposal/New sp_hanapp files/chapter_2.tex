%   Filename    : chapter_2.tex 
\chapter{Review of Related Literature}
\label{sec:relatedlit}

\section{Mobile Applications for Persons at Risk of Going Missing (PARGOM)}

As defined by the previous sections, any person that is experiencing trouble with navigating themselves can be categorized as a person at risk of going missing or PARGOM. This includes children, the elderly, or anyone experiencing medical, mental, or cognitive conditions that inhibit their ability to safely navigate back to their family or domicile. 
\\\\A mobile alert application determined to engage community volunteers to help in locating missing persons with dementia called the “Community ASAP system” was developed and documented in a paper by Neubauer, N., Daum, C., Miguel-Cruz, A., and Liu, L.. The findings of the study’s simulation of the Community ASAP system highlighted the importance of police services in these cases on account of their primary and direct involvement, and the effectiveness of community response and participation to the occurrence of missing persons with dementia. The approach also proves to be viable even for people who have no social media which is popular for being an accessible medium to disseminate information regarding MPs \cite{neubauer2021mobile}.
\\\\“CoSMiC”, a mobile application designed to crowdsource information about lost and missing children in situ is another approach towards applying techniques and technologies in locating persons at risk of going missing. The main concern that the CoSMiC mobile application prioritizes to solve is with regards to the urgency and criticality that ensues whenever a child goes missing within a neighborhood. The application aims to digitize crowdsourcing for finding missing children through a landmark-based location history of the lost child that was chronologically and locationally procured within the network of crowdsourced information \cite{shin2014cosmic}.
\\\\The urgency, reasoning, and crowdsourcing proposition and concerns by the above studies also reflect one of the main aims of the application being proposed; the care, prioritization, and the adaptation of approach towards PARGOMs.


\section{Serverless Application Model}

Serverless does not mean “server-less”. Serverless Computing is an up-and-coming paradigm and framework for the development and deployment of multiple applications, now relying on services through the cloud. A recent shift of enterprise applications’ architectures into containers, microservices, and serverless backend services has pushed the paradigm even more. Serverless platforms offer new features that make creating scalable microservices and applications easier and more cost efficient, promoting themselves as the next stage in cloud computing architectural evolution \cite{castro2017serverless}. One example of an application development software used in a serverless framework is Google’s Firebase, an example of BaaS (Backend-as-a-Service). \\
\\The proposed application also utilizes serverless computing as its paradigm. Knowing the general scenarios, perception, and support towards this new paradigm in software development is essential as it helps in the factoring and decision-making of the developers during the development of the proposed application.

\subsection{Google Firebase}

As defined by Khawas, C. and Shah, P. in their study titled “Application of Firebase in Android App Development-A Study” (2018), Firebase is one of the relatively new and even faster approaches towards handling large amounts of unstructured data as compared to the traditional Relational Database Management Systems (RDBMS) through developing serverless applications 
\cite{khawas2018application}.
\\\\In a paper by Hannula, T. (2021) titled “Unity mobile application with a serverless Firebase backend”, where a lo-fi themed Android mobile application prototype was developed, both NoSQL database services that are Firebase Realtime Database and Cloud Firestore were both utilized in managing the backend of the application the author has developed. Both services have enabled the application to be simplified and streamlined as Firebase handles the maintenance of the data in the Realtime Database and Cloud Storage \cite{hannula2021unity}.
\\\\As stated by the definition and example above, Google Firebase, an approach for serverless computing, is a promising choice for the chosen paradigm of the proposed application, serverless. Knowing the benefits and how Firebase was implemented on mobile applications is crucial in the development of the proposed application.


\section{Global Positioning System and GPS Applications}

\\The Global Positioning System (GPS) is, in itself, a United States-owned service that offers positioning, navigation, & timing (PNT) services to its users. As it is free, open, and reliable, GPS has been used and integrated countless times on a myriad of applications, including ones that are implemented in mobile platforms \cite{gpsGov}.
\\\\A seamless application of the Global Positioning System to Android mobile phones was implemented in a paper titled “Abhaya: An Android App for the Safety of Women" by Yarrabothu, R. S., and Thota, B. (2015). The paper describes an app called “Abhaya”, where it employs a quick and easy-to-use alert button from the application integrated within the smartphone, in which a single tap can identify the location or place of the user through the use of GPS services and would thus then send a message comprising the location URL to all registered contacts of the user. Additionally, Abhaya also calls the first person in the registered contact list of the user and periodically sends an SMS every five minutes to the registered contacts until the stop button has been tapped \cite{yarrabothu2015abhaya}.
\\\\A paper titled “Mobile phone application for reporting and tracking missing persons in Kenya ”, by Elizabeth Mutisya for the Strathmore University (2017), describes developing a centralized system and a mobile application, that hopes to organize the rather inefficient process in Kenya with regards to addressing missing persons cases. The said application uses GPS to determine and tag the location of a missing person, given that missing person has the application on hand, as well as of the sightings reported to increase the effectiveness of the app. It also utilizes the currently centralized database, National Missing and Unidentified Persons System (NamUs) in Kenya to both counter-check, verify, and assist in finding these said missing persons. A web application was also developed in order to assist those who would still want to access the software but do not have the smartphone needed for it to run the application (Android). The application was developed incrementally, starting on a smaller scale and gradually increasing in complexity using the Agile methodology \cite{mutisya2017mobile}.
\\\\Mutisya, E. (2017) has proposed the utilization of the Global Positioning System (GPS) in the assistance in providing more details about the disappearance of a missing person. This approach has exemplified the fact that GPS services can be applied in finding missing persons. Not to mention, the development methodology used in the paper is Agile as well, which is similar to the proposed application’s methodology, too. The Abhaya application \cite{yarrabothu2015abhaya} eventually provided the option for continuous tracking of the user or the missing person, provided that he or she has enabled the GPS in the proposed application, as periodically sending SMS instead of real-time tracking requires less processing power compared to the latter.
\\\\Overall, the utilization of the Global Positioning System (GPS) has provided an exemplary mechanism for tracking down missing persons by either identifying the last location the person was found or a real-time continuous tracking. This employs heightened identification of an individual’s whereabouts thus expediting the searching process.

\subsection{Google Maps Platform}

There have been numerous online mapping services available to the public, arguably the most popular of which is Google Maps. Google Maps, a web mapping service by Google, provides satellite imagery, street maps, real-time traffic conditions, and also route planning \cite{antony_2021}. Its closest competitor, Bing Maps, was developed by Microsoft, and a study showed that both provided near-similar accuracy in geocoding \cite{kilic2020accuracy}. However, due to the fact that Google Maps is more feature-rich, and has over 1 billion active users monthly — which translates to frequent location mapping and verification, Google Maps is considered to be the better option \cite{lookingbill2019google}.
\\\\Furthermore, Google has made it possible for developers to integrate and utilize Google Maps in the development of applications that require maps and location data. This is offered through the Google Maps Platform which is a collection of APIs, SDKs, and tools to easily embed and allow data retrieval from Google Maps to applications \cite{googleDevelopers}. In fact, there have been numerous studies showing the application of Google Maps and their open APIs. 
\\\\In Ghana, Google Maps together with GPS was used to successfully map accurate digital postal addresses to a separate application (GhanaPostGPS) by overlaying it over Google Maps through their API \cite{gah2018using}. Another study utilized GPS, Google Maps, and GSM technology (for texting location details) for parents and school authorities to accurately monitor children’s location in a timely manner which could be valuable if they go missing \cite{sunehra2016children}.
\\\\Google Maps and the Google Maps Platform’s open APIs and SDKs, together with GPS technology, makes it a viable technology to be used for developing apps for locating MPs in the country.


\subsection{Location based notification and alarm applications}

Technology and tools such as GPS and Google Maps (and the Google Maps Platform APIs), has made it possible to create applications to notify or alarm the user when they’re at (or near) certain locations. In 2013, an Android application was developed using Google Places API (offered in the Google Maps Platform) and GPS called “GEO ALERT” which would alert a traveler when they’re at a certain spot and it would show them a history of that specific location \cite{garg2013geo}. Another study proposed a location-based notification and alarm Android application utilizing GPS and LBS (location-based service), and it notified the user if a friend is nearby and alarms when the user enters a marked location on the map \cite{kanfade2018location}. More recently in 2022, a study titled “Travellert: A Location Based Alarm Application” developed an application using GPS and Google Maps Platform APIs and GPS which alerted a traveler when they were near their destination to avoid missing their stops \cite{travellert}.
\\\\The existence of these studies and location-based notification and alarm applications do not only provide proof of feasibility in using GPS, Google Maps, and similar technologies for the proposed application to track PARGOMs, but could serve as a basis on how to approach the development of the app’s location-based notification system for alerting users of PNP-verified MP cases near them.


\section{Other Missing Persons finder mobile applications}

The paper written by Desale H., Tavasalkar P., Vare S., and Shintre R. titled “Android Crime reporter and Missing Person Finder”, details some key insights that can be extremely helpful for the development of the project \cite{desale2020android}.
\\\\What is ingenious about this application is that the idea of having a “panic button" within the application can be a very useful feature. It can both limit the delays in relying on another user to notify the police about another user's becoming missing by having the said user notify the police right away. One key difference with this application compared to the proposed application, however, is that this only sends case reports and notifications directly to the police with the sole purpose of expediting the filing of these cases, whereas on the other hand, the team’s proposed application’s main goal is dissemination of information with regards to the missing person.
\\\\Studying the paper as a reference is worthwhile, as it was written with straightforwardness in the implementation of its functionalities in mind. The said authors wrote about the integration of a "panic button" in the mobile application, which, actually, has inspired the researchers to come up with a version of it as well in the proposed companion version of the application.
