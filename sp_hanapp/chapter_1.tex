%   Filename    : chapter_1.tex 
\chapter{Introduction}
\label{sec:researchdesc}    %labels help you reference sections of your document

\section{Overview of the Current State of Technology}
\label{sec:overview}
As defined by the International Commission on Missing Persons, subjectively, anyone who is being sought by at least another person and whose location or whereabouts are unknown can be classified as a missing person \cite{icmpMissing}. However, each country has their own standard or policies for legally defining a "missing" person, and as such, accurate statistics on the average rate of the disappeared globally are harder to comprehend. This, together with the unconfirmed number of the unreported cases of missing persons, people who voluntarily go missing, or even victims of disasters and conflict, brings to light how obscure and challenging a search for a missing person can be.

With this in mind, there have been some efforts globally to implement technologies and systems for organizing and solving this problem. The AMBER (America's Missing Broadcast Emergency Response) alert system, while not a standalone application, is a system that geo-locates and uses various forms of effective media such as smartphones, television, or radio to disseminate information about missing or abducted children in the United States in order to elicit citizen tips and responses in order to expedite the rescue of the said missing or abducted children \cite{griffin2007preliminary}.

NamUs, or the National Missing and Unidentified Persons System in the United States, is an integrated set of two databases: one for the information about unrecorded persons whose remains are inside the United States, and another for all profiles of missing persons and their information. Being publicly accessible for all entries and searches, the NamUs database has contributed to multiple 
projects, studies, works, and investigations for law enforcement \cite{murray2018history}. 

A push for a national Missing and Found Person Database (MFPD) in a 2016 Memorandum Circular from the National Police Commission in the Philippines has defined the database as a repository of all the names and relevant information about reported missing and found persons in the country. A website called the “Missing Person Website” was also defined in the memorandum, with the purpose of posting and displaying the name, picture, and other relevant information about missing and found persons 
\cite{NationalPoliceCommission}. The said database, however, is not open to the public unlike the NamUs. With further probing and scouring, the researchers have found that there is no clear, standardized, and modernized level of technology applied in the filing, dissemination, and searching for missing persons in the Philippines.

Currently, the standard procedure for filing missing persons (MP) cases in the Philippines is through the desk officers in their respective local police offices. This procedure can be both sluggish and counter-intuitive, especially in cases in which the missing person is a child or a victim of an accident, where the recording, investigation, and monitoring of the case needs to begin immediately \cite{NationalPoliceCommission}.

The researchers want to create an application that leverages current technology to streamline the processes required in the filing of missing persons cases and follows a similar formula to the AMBER alert system, which uses geo-location not only in the targeting of alerts but also in the tagging of the missing person's location, and through which it can both incite community levels of cooperation as well as dissemination of PNP-verified missing persons reports. 


\section{Problem Statement}
The Philippine Statistics Authority (2021) have only reported missing persons statistics for cases involving natural and man-made disasters: 5,000 MPs in 2020 and 12,000 MPs in 2021 \cite{PSAOpenStat}. On the other hand, the Philippine News Agency (2021) shared data from the DOJ which reported over 1.2 million reports of missing and/or exploited children in 2020, and 2.8 million reports in 2021 \cite{pulta_2021}. These, unfortunately, are only among the few publicly available numbers provided by national agencies with regards to MP cases in the Philippines.

Orion Support Incorporated (2015) estimated 35,000 MP reports each year in the country. With this, it relatively coincides with the rate of 1.7 persons per 1,000 Filipinos \cite{orion_2021}. However, the obscurity of reporting these missing cases has troubled not only the community but also authorities to craft effective measures in finding them in a short span of time — mitigating any factors that will further put the individual in danger especially if they are part or is a person-at-risk-of-going-missing (PARGOM).

The dire situation of these MP cases causes distress to their beloved in which our culture is deeply ingrained to familiarity and closeness of kinship or other relationships in the same nature.  Moreover, unresolved MP cases in the Philippines have drastic effects on the socio-economic health of the community, affecting not only the MP’s immediate family but the missing individual’s network and area. As MP cases remain unresolved, the sense of security in the area as well as the competence of local authorities are put into question, causing friction and restrictions. 

Although the PNP adopted MC 2016-033 which established a unified Missing and Found Persons Database (MFPD) and the process for receiving and handling MP reports in order to handle this issue, these guidelines stipulated that MP case reporting can only be done in person in the station of the area where the disappearance occurred \cite{NationalPoliceCommission}. Requiring the MP case reportee to travel and file the case in-person takes away valuable time, and neglects to utilize existing technology to streamline the process and have them receive, verify, and act on the case immediately. Moreover, the PNP has advised the public against posting and sharing of unverified MP cases on social media \cite{madarang_2022}.

As such, the researchers aim to create an application that leverages the use of existing technology to streamline and accelerate the reporting and verification process, allow users to be notified of PNP-verified missing persons cases in their area, track the location of loved ones whom they consider as persons-at-risk-of-going-missing (PARGOM), and assist PARGOMs in getting help when lost.



\section{Research Objectives}
\label{sec:researchobjectives}

\subsection{General Objective}
\label{sec:generalobjective}

The general objective of this study is to develop a mobile application that has multiple interfaces aimed at multiple target users, such as parents, their children, family members who are at risk of going missing (i.e. children or the elderly with dementia), the PNP, and many others, all in order to help file, verify, disseminate information, and know the whereabouts of missing persons.

\subsection{Specific Objectives}
\label{sec:specificobjectives}

This study specifically seeks:

\begin{enumerate}
   \item To develop an application that would allow users to easily report missing persons to the PNP and receive updates on such reports, as well as a PNP counterpart application that will enable the PNP to quickly receive, verify, manage, and respond to reports.
   \item To integrate a serverless database system for PNP-verified reports that enables location-based notifications so that app users can view information about missing persons nearby so they can assist in the search.
   \item To integrate location and global position system (GPS) services so that users can check on the last-known location of a PARGOM, provided that the PARGOM has linked their account to the user’s.
   \item To develop the companion application with a simpler interface in which users who are identified as PARGOMs can link their account with another user (e.g. child-parent) for simple and straightforward location sharing between said users.
\end{enumerate}

\section{ Definition of Terms and Acronyms}

\textbf{Child} - refers to any person under the age of 18 \\ \\ 
\textbf{Database} - also referred to as \textbf{“serverless database”}, refers to the application’s own public database containing the reports filed to the PNP pending their verification, the PNP-verified missing persons reports which will be used to notify nearby users of missing persons cases in their area \\ \\
\textbf{HanApp} - also referred to as “the application” or “proposed application”, will refer to the application and its various interfaces being developed in this study. Interfaces:
\begin{itemize}
    \item \textbf{Main (or parent) App} - refers to the main application that has the features: reporting and getting updates on missing persons cases, locating the user of the companion app
    \item \textbf{Companion App} - refers to the limited-feature application interface to be used by persons at risk of going missing (e.g. a child or a person with dementia)
    \item \textbf{PNP Admin App} - refers to the application interface that is accessed by the PNP in which they receive, manage, verify, and provide updates on filed reports
\end{itemize}
\textbf{Located} - will be used to denote that the missing person has been identified and reunited with their family or guardian.\\\\
\textbf{Missing Persons (MP)} - will refer to any person who is classified under the PNP guidelines as missing; an adult person who is missing is required under the guidelines to have not been located after 24 hours, whereas a child that has gone missing is immediately referred to as an MP.\\\\
\textbf{Missing and Found Person Database (MFPD)} - refers to the PNP’s own private database where they manage Missing and Found Persons cases, and will not be accessible by the application.\\\\
\textbf{Person-at-risk-of-going-missing (PARGOM)} - refers to any person that the main app user deems to be at risk of going missing and is unable to find their way back, such as a child or an elderly person with dementia. If an adult PARGOM has not been located, they are technically not classified by the PNP as MP until 24 hours from perceived disappearance.\\\\
\textbf{Reportee} - refers to the person filing the missing person report via the application.

\section{Scope and Limitations of the Research}
\label{sec:scopelimitations}

The application primarily aims to streamline and hasten the filing and verification of missing persons reports to the PNP and prevent the dissemination of unverified missing persons reports which the PNP advises against.
\\\\The application, at least for its current proposed build, mainly focuses on the filing and locating of missing persons in scenarios not involving post-disaster search and relief operations (i.e. after an earthquake, typhoon, landslide).
\\\\Moreover, this application utilizes a serverless approach with the integration of Firebase, not to mention, services like Google Maps and other location based services all rely on stable internet connectivity, be it through mobile data or Wi-Fi.
\\\\Finally, in order to facilitate information dissemination of PNP-verified missing persons reports, and to notify and allow the community in assisting with the search for missing persons in their area, a public serverless database will be created storing the PNP-verified missing persons reports. It is important to note that this is a separate database from the PNP’s own “Missing and Found Persons” database.
\subsection{User and Information Access}
The application will only have three interfaces which will cater to its main three users: the general user, the PNP personnel/helpdesk in charge of missing persons cases, and persons at risk of going missing (PARGOMs). Since this is a mobile application (with a desktop interface for the PNP) and the scope of the application is only national, only users in the Philippines with compatible smartphones and access to the internet will be able to use the app.
\\\\Due to the sensitivity of the data being handled, it’s important to define the scope and limits of each user’s access to address data privacy and security concerns.
\begin{enumerate}
    \item The general users refer to the general public who either wish to report a missing persons case, assist with search efforts for missing persons, or wish to keep track of and locate a loved one who is at risk of going missing. They will only have access to information regarding their registered account, the updates provided by PNP on the report they filed, PNP-verified missing persons in their area, and the location information of the PARGOMs linked to their account.
    \item The PNP personnel/helpdesk in charge of missing persons cases and have access to their HanApp PNP account will only have access to information provided through reports, and will not be given access to any other information from the users.
    \item The PARGOMs users are any persons who are at risk of going missing and may not have the capability to find their way back (e.g. children or those with cognitive disabilities such as dementia). They will only have access to their account information, their location, and the directory and location of the nearest PNP station or helpdesk.
\end{enumerate}
Moreover, the application’s administrators and developers will only have access to the serverless database containing the list of verified missing persons cases and their locations, and the account names at most, like user passwords, and location information will not be accessible by anyone apart from the users themselves. 
\\\\The PNP’s own private “Missing and Found Person Database” and all information therein will be separate from the serverless database to be used for storing verified missing persons cases, and will not be accessed by the application, the application’s users, nor its administrators.


\section{Significance of the Research}
\label{sec:significance}
The research’s suggested mobile application and its framework aim to consolidate and streamline the process of filing and verifying missing persons reports, and also allow for an efficient way to disseminate PNP-verified missing persons reports. Thus, accelerating the process of resolving missing persons cases through quicker report verification, location-based alerts, and leveraging community-based search efforts. 
\\\\This application and framework will be of great significance for protective services and individuals whose loved ones have gone missing by allowing them to report and locate the whereabouts of these missing individuals with ease.
\\\\In addition, the data that will be borne out of this research will be a pivotal factor in current events and statistics relating to missing persons. In fact, the focal point of this research paves the way for a systematic and technological breakthrough here in the Philippines to create a mobile application that will positively influence resolving missing person cases. Unequivocally, the novelty of this study provides a baseline framework for similar studies impacting the creation of a more robust system in the near future.


